% ----------------------------------------------------------------------------------------%
%	Created by Alessandro with TeXShop						%
%	---->	May 27, 2009										%
%	Compiled with XeLaTeX, on Mac OS X						%
%	Licensed under the Creative Commons Attribution 3.0 Unported	%
%	Share, change, spread, and have fun!						%
%	http://creativecommons.org/licenses/by/3.0/					%
%	You can find more at http://aleplasmati.comuv.com				%
% ----------------------------------------------------------------------------------------%
%  Copyright 2011 Jason Filley
%  I started from Alessandro's template at http://www.cv-templates.info/
%  Multi-page resumes annoy me -- you're too wordy, or 
%  you're listing every product/knowledge you've ever used (if only for 
%  an hour) or overheard someone talking about.  Get to the point.
%
%  http://www.snakelegs.org
%
%  Only tested with XeTeX.  http://scripts.sil.org/xetex
%
%  Same license as above, of course.
%
%  Minor caveat: in the US, at least, employees frequently demand a Microsoft 
%  Word document version of this, but a straight copy/paste loses the small caps.  
%  There are online PDF to Word converters (Rich Text Format), but you may have to 
%  retype a bit, like the section titles.  But it really looks nice as a printed PDF....
% ----------------------------------------------------------------------------------------%

% adjust the points
\documentclass[a4paper,fontsize=10pt]{scrartcl} %10pt when you add more

%\pdfpagewidth 8.5in
%\pdfpageheight 11in 

\usepackage{polyglossia}
\setdefaultlanguage{portuges}


\setlength\topmargin{0in}
\setlength\headheight{0in}
\setlength\headsep{0in}
\setlength\textheight{10in}
\setlength\textwidth{6in}
\setlength\oddsidemargin{0in}
\setlength\evensidemargin{0in}
 
%Colors/Graphics
\usepackage{color,graphicx,multicol}
\usepackage[usenames,dvipsnames]{xcolor}
\usepackage{pifont}   % for dingbats
\usepackage[hmargin=.5cm, vmargin=.5cm]{geometry}               
\usepackage{paralist} % for inparaitem to squish lists

% blue dingbat bullets in address
\definecolor{bullets}{HTML}{104170}
\newcommand{\mybullet}{\mbox{\LARGE\raisebox{+0.2ex}{\tiny{\ding{118}}}}}
\renewcommand{\labelitemi}{\textcolor{bullets}{\mybullet}}

%Fonts and Tweaks for XeLaTeX
\usepackage{fontspec,xltxtra,xunicode,textcomp}
\defaultfontfeatures{Mapping=tex-text}
\setromanfont[Ligatures={Common,Contextual}]{Calluna}        %http://www.josbuivenga.demon.nl/calluna.html
	% n.b.: no italics in free version of Calluna / regular face only
\setsansfont[Ligatures={Common,Contextual,Rare}]{Trajan Pro}    %http://castletype.com/html/tipoteca/goudy-trajan-regular.html
\setmonofont[Scale=MatchLowercase,Ligatures={Common,Contextual,Rare}]{Inconsolata} %http://www.levien.com/type/myfonts/inconsolata.html
%Setup hyperref package, and colours for links, text and headings
\usepackage[pdftitle={Curriculum Vitae de Vítor Matias},pdfauthor={Vítor Matias}]{hyperref}
\definecolor{linkcolour}{HTML}{000000}	% change if you intend for it to be read
	% in a viewer, as opposed to being printed.
\definecolor{text1}{HTML}{000000}		
\definecolor{headings}{HTML}{701112} 	% 701112 dark red

\hypersetup{colorlinks,breaklinks,
			urlcolor=linkcolour, 
			linkcolor=linkcolour}

\usepackage{fancyhdr}				%custom footer
\pagestyle{fancy}
\fancyhf{}
%\rfoot{\color{headings}
%{\sffamily Copyright \copyright~2011 Your Name. 
%	All rights reserved.}}
	
\renewcommand{\headrulewidth}{0pt}
\usepackage{titlesec}				%custom \section
\setlength{\columnsep}{0.04\textwidth}
\titleformat{\section}
	{\color{headings}
		\scshape\Large\raggedright}{}{0em}{}[\color{black}\titlerule]
\titlespacing{\section}{0pt}{0pt}{5pt}

\begin{document}
\color{text1} % set text color for the whole doc
\center{\sffamily\Huge\color{headings} Vítor Daniel de Oliveira Matias}\\
\textsc{\large{Rua de Trás os Montes, n\textordmasculine\ 26, 2835-099 Baixa da Banheira
{\tiny{\textcolor{bullets}{\ding{70}}}} +351 96 368 4227
{\tiny{\textcolor{bullets}{\ding{70}}}}
\href{mailto:vitor@xorbitor.com,vitor.o.matias@gmail.com}{vitor@xorbitor.com}}}\\

\color{text1}

%START of left-hand side minipage
\begin{minipage}[t]{0.5\textwidth} 
\vspace{0pt}	%trick

\section{Experiência Profissional e Académica}

% use empty [] after the first item to suppress the bullet
% \textsc{Instituto Nacional de Estatística --- Recenseador}\\
% \small\textsc{Mar~2011 até Abr~2011 -- Lavradio, Barreiro, Portugal}\\ \normalsize
% \begin{inparaitem} 
% 	\item[] Responsável pela distribuição e recolha de questionários
% 	\item Auxilio à população no preenchimento dos questionários
% 	\item Contagem e verifica\c{c}ão de questionários.
% \end{inparaitem}
% \\\par

\textsc{(Nokia Siemens Networks) Coriant --- Software Developer}\\
\small\textsc{Nov~2012 até Agora -- Alfragide, Amadora, Portugal}\\ \normalsize
Parte da equipa de front end da plataforma hiT 7300.\\
\begin{inparaitem} 
%	\item[] Análise de Requisitos da Aplicação
	\item[] Análise e resolu\c{c}ão de problemas reportados por testers/clientes
	\item Desenvolvimento de janelas em Java Swing
	\item Extensão e manuten\c{c}ão de uma biblioteca baseada na SwingLibrary para a RobotFramework
	\item Elaboração de testes de GUI com a RobotFramework
	\item Gestão da plataforma de Continous Integration da equipa (Jenkins)
\end{inparaitem}
\\\par


\textsc{FCT/UNL --- Aluno}\\
\small\textsc{Mar~2012 até Jul~2012 -- Caparica, Almada, Portugal}\\ \normalsize
Na edição de 2012 da Cadeira de Projeto Integrador foi pedido aos alunos que desenvolvessem uma aplicação para plataformas móveis que melhorasse a experiência de visitantes de eventos.\\
\begin{inparaitem} 
	\item[] Análise de Requisitos da Aplicação
	\item Desenvolvimento da Aplicação em Ruby on Rails (servidor), HTML5 e JQuery Mobile (cliente)
	%\item Teste e correcções da Aplicação
	\item Elaboração de um poster e workshop.
\end{inparaitem}
\\\par

\textsc{FCT/UNL --- Aluno}\\
\small\textsc{Set~2011 até Dez~2011 -- Caparica, Almada, Portugal}\\ \normalsize
Trabalho final da cadeira de Engenharia de Software.
Modelação e implementação de um protótipo de uma rede social.\\
\begin{inparaitem} 
	\item[] Elicitação de requisitos utilizando Viewpoints e Concerns
	\item Elaboração de diagramas de Casos de Uso, Sequência, Classes e Features 
	\item Modelação do sistema com Notação Z
	\item Elaboração de um diagrama de Pacotes e de Instalação
	\item Implementação de um protótipo funcional em Ruby on Rails.
\end{inparaitem}
\\\par

\textsc{FCT/UNL --- Aluno}\\
\small\textsc{Set~2010 até Dez~2010 -- Caparica, Almada, Portugal}\\ \normalsize
Trabalho final da cadeira de Sistemas Distribuídos.
O objetivo era implementar uma aplicação que permitisse sincronizar e partilhar ficheiros entre diversas máquinas e utilizadores em Java.\\
\begin{inparaitem} 
	\item[] Implementação da árvore de ficheiros virtual
	\item Implementação de comunicações recorrendo a Remote Method Invocation (Java RMI)
	\item Implementação de comunicações seguras com pares de chaves públicas e privadas
	\item Implementação de autenticação segura usando o algoritmo Needham-Schroeder.
\end{inparaitem}
\\\par

\textsc{FCT/UNL --- Aluno}\\
\small\textsc{Mar~2010 até Jun~2010 -- Caparica, Almada, Portugal}\\ \normalsize
Trabalho final da cadeira de Métodos de Desenvolvimento de Software.
O objetivo deste trabalho foi modelar um sistema de apoio ao socorro de acidentes rodoviários.\\
\begin{inparaitem}
	\item[] Identificação e descrição do problema
	\item Elaboração de um diagrama de Casos de Uso e descrição dos mesmos com recurso a vários cenários
	\item Elaboração de diagramas de Actividade, Sequência e Estado de quatro casos de uso
	\item Elaboração de um diagrama de Classes
	\item Criação de restrições com Object Constraint Language
	\item Elaboração do diagrama de Componentes e Pacotes do sistema.
\end{inparaitem}
\\\par

\textsc{FCT/UNL --- Aluno}\\
\small\textsc{Set~2008 até Dez~2008 -- Caparica, Almada, Portugal}\\ \normalsize
Trabalho final da cadeira de Base de Dados. Uma Base de Dados de uma Coletividade em Oracle 10g e Oracle Forms.\\
\begin{inparaitem} 
	\item[] Elaboração de um Diagrama Entidade-Relação
	\item Criação do Esquema de Tabelas em SQL
	\item Implementação de Vistas e Forms para os diferentes tipos de utilizadores da plataforma
%	\item Implementação de diversos Triggers.
\end{inparaitem}
\\\par

% \textsc{FormaDeSer --- Estagiário (Técnico de Informática)}\\
% \small\textsc{Jan~2007 to Abr~2007 -- Barreiro, Portugal}\\ \normalsize
% \begin{inparaitem}
% 	\item[] Reparação e substituição de computadores
% 	\item Listagem das máquinas da firma
% 	\item Cria\c{c}ão de uma base de dados em Microsoft Access.
% \end{inparaitem}
% \\\par
% % \textsc{Acme Logistics --- (Senior Bean Counter)}\\
% \small\textsc{May~2005 até  April~2007 -- City, ST}\\ \normalsize
% \begin{inparaitem}
% 	\item[] Duis massa odio, tincidunt sit amet vehicula quis, hendrerit vitae 
% 		felis. Vestibulum ac diam vel arcu porttitor varius vel ac erat.
% 	\item Ut in dui ac nunc auctor dignissim eu aliquam justo. In hac habitasse platea dictumst.
% 	\item Proin ipsum justo, elementum eu imperdiet at, cursus non nibh. Vivamus 
% 		aliquam rhoncus nunc, porta pretium risus congue sit amet.
% \end{inparaitem}
% \\\par
% use empty [] after the first item to suppress the bullet


\end{minipage} %END of left-hand side minipage
\hfill
\begin{minipage}[t]{0.46\textwidth} %START of right-hand side minipage

\vspace{0pt} %trick for alignment
	
\section{Proficiências Técnicas}
% separate each type with a bullet
% give each initial item an emtpy parameter (e.g., \item[]) to suppress 
%   the bullet
\begin{inparaitem}
\item[] Debian Linux, Ubuntu Linux, CentOS, Arch Linux,  Microsoft Windows
\item Java, C\#, CIL, C, C++, OCAML, Ruby, Python, PHP, \TeX, JavaScript, Unix Shell Scriping
\item HTML, \LaTeX, \XeTeX, \XeLaTeX\
\item Eclipse, MonoDevelop, NetBeans, Aptana Studio, Kile
\item Ruby on Rails, Play Framework, jQuery, jQuery Mobile
\item Subversion, Git, Mercurial
\item GCC, Mono, Java SE, RobotFramework 
\item Apache, lighttpd, Nginx
\item MySQL, PostgreSQL, Oracle, SQLite
\item UML, Nota\c{c}ão Z
\end{inparaitem}\\\par
\section{Forma\c{c}ão Académica e Certifica\c{c}ões}
\begin{inparaitem}
\item[] Atualmente a frequentar a Licenciatura em Engenharia Informática na Faculdade de Ciências e Tecnologia da Universidade Nova de Lisboa (12,512 valores, 162 de 180 ECTS's obtidos)
\item Curso Tecnológico de Informática pela Escola Secundária Augusto Cabrita (2007, 16,3 valores)
\item Nível Ouro, Prata e Bronze da FUTUREKIDS (2000, 1999 e 1998, respetivamente)
\item Carta de Condu\c{c}ão, categoria B
\end{inparaitem}
\\\par
\section{Línguas}
\textsc{Português --- Fluente}\\
\small\textsc{Língua materna.}
\\\par
\textsc{Inglês --- Avançado}\\
\small\textsc{Boa comunicação da língua escrita e falada.}
\\\par
%\vspace{0pt} %trick for alignment
\section{Eventos e Congressos}
\textsc{X Encontro Nacional sobre Tecnologia Aberta --- Sybase/Caixa Mágica}\\
\small\textsc{29~Set~2011 -- Fórum Tecnológico de Lisboa, Portugal}\\ \normalsize
 %\begin{inparaitem}
 %	\item[] Participação em palestras e sessões técnicas.
 %\end{inparaitem}
\\\par
\textsc{IX Encontro Nacional sobre Tecnologia Aberta --- Sybase/Caixa Mágica}\\
\small\textsc{29~Set~2011 -- Fórum Tecnológico de Lisboa, Portugal}\\ \normalsize
 %\begin{inparaitem}
 %	\item[] Participação em palestras e sessões técnicas.
 %\end{inparaitem}
\\\par
\textsc{VIII Encontro Nacional sobre Tecnologia Aberta --- Sybase/Caixa Mágica}\\
\small\textsc{30~Set~2010 -- Fórum Tecnológico de Lisboa, Portugal}\\ \normalsize
 %\begin{inparaitem}
 %	\item[] Participação em palestras e sessões técnicas.
 %\end{inparaitem}
\\\par
\textsc{VII Encontro Nacional sobre Tecnologia Aberta --- Sybase/Caixa Mágica}\\
\small\textsc{24~Set~2009 -- Fórum Tecnológico de Lisboa, Portugal}\\ \normalsize
 %\begin{inparaitem}
 %	\item[] Participação em palestras e sessões técnicas.
% \end{inparaitem}
\\\par
% \vspace{5pt}%REMOVER!!!
\textsc{Encontro Nacional do Colégio de Engenharia Informática --- Ordem dos Engenheiros}\\
\small\textsc{07~Set~2009 e 08~Set~2009 -- Instituto Superior Técnico no Taguspark, Oeiras, Portugal}\\ \normalsize
 %\begin{inparaitem}
  %	\item[] Participação em palestras e sessões técnicas.
 %\end{inparaitem}
\\\par
\textsc{VI Encontro Nacional sobre Tecnologia Aberta --- Sybase/Caixa Mágica}\\
\small\textsc{15~Abr~2008 -- Fórum Tecnológico de Lisboa, Portugal}\\ \normalsize
%\begin{inparaitem}
% 	\item[] Participação em palestras e sessões técnicas.
 %\end{inparaitem}
\\\par
\textsc{V Encontro Nacional sobre Tecnologia Aberta --- Sybase/Caixa Mágica}\\
\small\textsc{19~Abr~2007 -- Fórum Tecnológico de Lisboa, Portugal}\\ \normalsize
%\begin{inparaitem}
 %	\item[] Participação em palestras e sessões técnicas.
 %\end{inparaitem}
\\\par
\end{minipage} %END of right-hand side minipage
\end{document}  
